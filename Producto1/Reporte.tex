% Ejemplo de documento LaTeX
% Tipo de documento y tamaño de letra
\documentclass[12pt]{article}

% Preparando para documento en Español.
% Para documento en Inglés no hay que hacer esto.
\usepackage[spanish]{babel}
\selectlanguage{spanish}
\usepackage[utf8]{inputenc}
\usepackage{chngpage}
% EL titulo, autor y fecha del documento
\title{Tutorial breve de los comandos de Bash}
\author{Andr\'es Rodr\'iguez}
\date{}
\newcommand{\specialcell}[2][l]{%
  \begin{tabular}[#1]{@{}l@{}}#2\end{tabular}}
  
% Aqui comienza el cuerpo del documento
\begin{document}
% Construye el título
\maketitle

\section{¿Qué es {\tt bash}?}
Bash es un shell de Unix que es utilizado en el sistema operativo Linux. Es un procesador de comandos que corre tipicamente en una ventana de texto, donde el usuario escribe comandos que causan acciones. Bash tambi\'en lee comandos de un archivo, llamado script. Su funci\'on es la de medidor entre el usuario y el sistema. 
Bash es un interpretador de comandos utilizado sobre el sistema operativo Linux.
Su función es de mediar entre el usuario y el sistema.

\section{Comandos Bash}

\medskip% adds some space before the table
\begin{adjustwidth}{-1in}{-1in}% adjust the L and R margins by 1 inch
\centering
\begin{tabular}{|c|l|l|}
\hline
Comando & Descripción & Ejemplo \\ \hline
mkdir & Crea directorios & mkdir dir1\\ \hline
cp    & Copia ficheros &  cp dir1/subdir1/* dir2/\\ \hline
mv  &  Mueve ficheros. & mv archivoA.dat archivoX.dat\\ \hline
rm   & Elimina ficheros.	& rm archivo*\\ \hline
file  & Muestra el tipo de archivo.	& file mailbox\\ \hline
ls   & Lista contenidos de un directorio	& ~/dir1\$ ls \\ \hline
find & Encuentra archivos	& find arch \\ \hline
\end{tabular}

\begin{tabular}{|c|l|l|}
\hline
sudo & \specialcell{Permite ejecutar un comando\\ desde como otro usuario} &	sudo u root nano...\\ \hline
chown & Cambia el propietario de un fichero	&  chown R root  subdir1\\ \hline
echo & Muestra texto.			&			~/dir1\$ echo hola\\ \hline
cd   & Cambia de directorio.		&			~\$ cd dir1/\\ \hline
pwd &  Imprime la dirección del directorio actual.	&	~\$ pwd/home/home\\ \hline
chmod & Cambia los permisos de un fichero.		&	\specialcell{~/dir1\$ chmod u+x\\archivo.tar.gz}\\ \hline
tac  & Concatena e imprime archivos invertidos.		&	~/dir1\$ tac archivo1.dat\\ \hline
tr   & Traduce o borra caracteres		&		~/dir1\$ tr nh ñ\\ \hline
grep & \specialcell{Muestra las coincidencias con\\la entrada que se le brinde} & dpkg l grep klippe\\ \hline
man  & Permite acceder al manual del comando dado& man k texto a buscar\\ \hline
cal  & Muetra un calendario &  \\ \hline
cmp  & Compara dos filas &	cmp mifile1 mifile2\\ \hline
date & Muestra o cambia la fecha y hora & \\ \hline
dc   & Calculadora de escritorio & \\ \hline
eject & Expulsa media removible & \\ \hline
kill & Detiene procesos de estar corriendo	& kill -HUP <pid>\\ \hline
tail & \specialcell{Muestra las ‘n’ ultimas lineas de un\\ fichero texto}  &  tail ~/mifile\\ \hline
sleep & Retrasa por cierto tiempo	& sleep 5\\ \hline

\end{tabular} 
\end{adjustwidth}
\medskip% adds some space after the table



% Nunca debe faltar esta última linea.
\end{document}