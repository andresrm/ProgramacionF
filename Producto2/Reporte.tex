\documentclass{article}

\title{Compiladores e Interpretadores}
\author{Andr\'es Rodr\'iguez}
\date{}


\begin{document}

\maketitle

\section{Introducci\'on}
Un comprilador traduce completamente un programa fuente escrito en un lenguaje de alto nivel a un lenguaje ensamblador o maquina.
\section{Tabla comparativa}
\begin{tabular}{|c | c | c | c | c | c |}

\hline

Nombre & Paradigma & Creadores & A\~no & Extensiones & Compilacion \\ \hline \hline

C & Imperativo & Bell Labs & 1972 & .h .c & \\ \hline

C++ & Imperativo & Bjarne Stroustrup  & 1983 & .cpp \\ \hline

Fortran & Imperativo & IBM & 1957 & .f90\\ \hline

Java & Orientado a objetos & Sun Microsystems & 1995 & .java\\ \hline

Python & Funcional & Guido van Rossum & 1991 & .py\\ \hline

Ruby & O.O, reflexivo & Yukihiro Matsumoto & 1995 & .rb \\ \hline

\end{tabular}


\section{Ejemplo}


\end{document}