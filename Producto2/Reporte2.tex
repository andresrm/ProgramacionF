\documentclass{article}

\title{Compiladores e Interpretadores}
\author{Andr\'es Rodr\'iguez}
\date{}


\begin{document}

\maketitle

\section{Introducci\'on}
Un compilador traduce completamente un programa fuente escrito en un lenguaje de alto nivel a un lenguaje ensamblador o m\'aquina. Usualmente el segundo lenguaje es lenguaje de m\'aquina, pero tambi\'en puede ser un c\'odigo intermedio, o simplemente texto. Este proceso de traducci\'on se conoce como compilaci\'on.

\section{Tabla comparativa}
\begin{tabular}{|c | c | c | c | c|}

\hline
Nombre & Paradigma & Creadores & A\~no & Extensiones\\ \hline \hline

C & Imperativo & Bell Labs & 1972 & .h .c \\ \hline

C++ & Imperativo & Bjarne Stroustrup  & 1983 & .cpp \\ \hline

Fortran & Imperativo & IBM & 1957 & .f90\\ \hline

Java & Orientado a objetos & Sun Microsystems & 1995 & .java\\ \hline

Python & Funcional & Guido van Rossum & 1991 & .py\\ \hline

Ruby & Orientado a objetos, reflexivo & Yukihiro Matsumoto & 1995 & .rb \\ \hline

\end{tabular}


\section{Ejemplos}

\begin{itemize}
\item Fortran
\end{itemize}
\begin{verbatim}
program juego
   
 write(*,*) 'Hola! Trataré de adivinar un número'
    write(*,*) 'Piensa un número entre 1 y 10.'
    call sleep(5)
    write(*,*) 'Ahora multiplícalo por 9.'
    call sleep(5)
    write(*,*) 'Si el número tiene 2 digítos, súmalos entre si; Si tu número tiene unn solo dígito, súmale 0.'
    call sleep(5)
    write(*,*) 'Al número resultante súmale 4'
    call sleep(10)
    write(*,*) 'Muy bien. El resultado es 13.' 

end program juego

\end{verbatim}

\begin{itemize}
\item C++
\end{itemize}

\begin{verbatim}
#include <iostream>
#include <Windows.h>

using namespace std;
int main()

{   
	cout << "Hola! Trataré de adivinar un número.";
	cout <<
	"Piensa un numero entre 1 y 10.";
	Sleep(5000);
	cout << "Ahora multiplícalo por 9.";
	Sleep(5000);
	cout << "Si el número tiene 2 dígitos, súmalos entre si: Ej. 36 -> 3+6=9. Si tu número tiene un solo dígito, súmale 0.";

	Sleep(5000);

	cout << "Al número resultante súmale 4.";

	Sleep(10000);

	cout << "Muy bien. El resultado es 13 :)";

	return 0;

}

\end{verbatim}

\begin{itemize}
\item Ruby
\end{itemize}

\begin{verbatim}

# -*- coding: utf-8 -*-

puts "Hola! Trataré de adivinar un número."

puts "Piensa un número entre 1 y 10."

sleep(5)  

puts "Ahora multiplícalo por 9."

sleep(5)

puts "Si el número tiene 2 dígitos, súmalos entre si: Ej. 36 -> 3+6=9. Si tu número tiene un solo dígito, súmale 0."

sleep(5) 

puts "Al número resultante súmale 4."

sleep(10) 

puts "Muy bien. El resultado es 13 :) "

\end{verbatim}

\begin{itemize}
\item Python
\end{itemize}

\begin{verbatim}
import time

print "Hola! Tratare de adivinar un numero."
import time
time.sleep(5)

print "Piensa un numero entre 1 y 10."
import time
time.sleep(5)

print "Multiplicalo por 9"
import time
time.sleep(5)

print "Si el numero es de dos digitos, sumalos entre si: Ej. 36  -> 3+6=9. Si tu numero tiene un solo digito, sumale 0"
import time
time.sleep(5)

print "Al numero resultante sumale 4"
import time
time.sleep(10)

print "Muy bien. El resultado es 13 :)"
print ""
import time
time.sleep(5)
import time
time.sleep(5)

\end{verbatim}

\begin{itemize}
\item Java
\end{itemize}

\begin{verbatim}

public class juego{
    public static void main(String[]args){
	System.out.println("Hola!, Tratare de adivinar un numero");


	try{
	    Thread.sleep(5000);
	}

	catch(InterruptedException ex){
	    Thread.currentThread().interrupt();
	}

	System.out.println("Multiplicalo por 9");

	try{
	    Thread.sleep(5000);
	}

	catch(InterruptedException ex){
	    Thread.currentThread().interrupt();
	}

	System.out.println("Si el numero es de dos digitos, sumalos entre si.");

	try{
	    Thread.sleep(5000);
	}

	catch(InterruptedException ex){
	    Thread.currentThread().interrupt();
	}

	System.out.println("Al numero resultante sumale 4 ");
	try{
	    Thread.sleep(10000);
	}

	catch(InterruptedException ex){
	    Thread.currentThread().interrupt();
	}
	
	System.out.println("Muy bien. El resultado es 13 :)");
    }
}
\end{verbatim}
\end{document}