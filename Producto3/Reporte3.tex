\documentclass{article}
\usepackage[export]{adjustbox}
\usepackage{graphicx}
\usepackage{fancyvrb}


\title{Reporte Actividad 3}
\author{Andr\'es Rodr\'iguez}
\date{}


\begin{document}

\maketitle

\graphicspath{ {Imagenes/} }


\section*{Introducci\'on}

El objetivo en esta pr\'actica es introducirnos al lenguaje de programaci\'on Fortran. El presente reporte detalla las actividades realizadas con este prop\'osito, se describe brevemente cada una, se a\~nade una imagen de la ejecuci\'on y se muestra el codigo correspondiente.
\section*{Actividades}

\subsection*{Area}
Calcular el \'area de un c\'irculo, especificando el radio durante la ejecuci\'on.\\ \\
\adjincludegraphics[scale=0.67,trim={0 0 {.5\width} 0},clip]{Area}
	
\begin{Verbatim}[frame=single]

Program Circlearea

Implicit None

Real :: radius , circum , area
Real :: PI = 4.0 * atan(1.0) 
Integer :: model_n = 1 

print *, 'Enter a radius:'
read *, radius 
circum = 2.0 * PI * radius
area = radius * radius * PI
print * , 'Program number =' , model_n
print * , 'Radius =' , radius
print * , 'Circumference =' , circum
print * , 'Area =' , area

End Program Circlearea


\end{Verbatim}

\subsection*{Volume}
Calcular el volumen de un l\'iquido en un tanque esf\'erico cuando el l\'iquido est\'a a una altura h del fondo del tanque. En la ejecuci\'on se define el radio y la altura h.\\ \\
\adjincludegraphics[scale=0.67,trim={0 0 {.5\width} 0},clip]{Volume}


\begin{Verbatim}[frame=single]

Program Volume

Implicit NONE

Real :: radius ,h, vol
Real :: PI = 4.0 * atan(1.0) 
Integer :: model_n = 1 

print *, 'Enter a radius:'
read *, radius 

print *, 'Enter a height:'
read *, h


vol = PI *h*h*(3*radius - h ) / 3

print * , 'Program number =' , model_n
print * , 'Radius =' , radius
print * , 'Volume =' , vol

End Program Volume

\end{Verbatim}


\subsection*{Limits}
El programa determina la precisi\'on de la m\'aquina. Se compara repetidamente $1 + \epsilon_m$ con 1 a medida que $\epsilon_m$ se vuelve mas peque\~no y se muestra como impacta en la precisi\'on del c\'alculo.\\ \\
\adjincludegraphics[scale=0.67,trim={0 0 {.5\width} 0},clip]{Limits}
	
\begin{Verbatim}[frame=single]
Program Limits
  Implicit None
  Integer :: i, n
  Real*8 :: epsilon_m, one
  n=60         

  epsilon_m = 1.0
  one = 1.0


  do  i = 1, n, 1                 
    epsilon_m = epsilon_m / 2.0  
    one = 1.0 + epsilon_m         
    print *, i, one, epsilon_m    
  end do                          

End Program Limits
\end{Verbatim}


\subsection*{Limits 4}
Se modifica el programa anterior cambiado la precisi\'on a sencilla.\\ \\
\adjincludegraphics[scale=0.67,trim={0 0 {.5\width} 0},clip]{Limits4}
	
		
\begin{Verbatim}[frame=single]
Program Limits
  Implicit None
  Integer :: i, n
  Real*4 :: epsilon_m, one
  n=60         

  epsilon_m = 1.0
  one = 1.0


  do  i = 1, n, 1                 
    epsilon_m = epsilon_m / 2.0  
    one = 1.0 + epsilon_m         
    print *, i, one, epsilon_m    
  end do                          

End Program Limits


\end{Verbatim}


\subsection*{Mathtest}
Se muestra el uso de algunas funciones matematicas en Fortran y se imprimen en la pantalla de terminal los ejemplos de sus valores.\\ \\
\adjincludegraphics[scale=0.67,trim={0 0 {.5\width} 0},clip]{Math}

	
\begin{Verbatim}[frame=single]
Program Mathtest
  
  Real *8 :: x=1.0 , y, z 
  y=sin(x)
  z=exp(x) + 1.0
  
  Print * , x, y, z

End Program Mathtest
\end{Verbatim}


\subsection*{Mathtest2}
Se expone el uso de variables complejas para obtener los valores que se pide.\\ \\
\adjincludegraphics[scale=0.67,trim={0 0 {.5\width} 0},clip]{Math2}
	
	\adjincludegraphics[scale=0.67,trim={{.5\width} 0 0 0},clip]{Math2}
	
		
\begin{Verbatim}[frame=single]
Program Mathtest2
  
  COMPLEX, PARAMETER    :: MINUS_ONE = -1.0, x= 2.0
  COMPLEX               :: Z,y
  Real *8 :: w, q=1.0

  Z = SQRT(MINUS_ONE)
  y=asin(x)
  w=log(q-1)

  
  Print * , Z, y, w

End Program Mathtest2

\end{Verbatim}
	
\subsection*{Function}
Aqu\'i se muestra el uso de funciones definidas en el c\'odigo.\\ \\
\adjincludegraphics[scale=0.67,trim={0 0 {.5\width} 0},clip]{Function}
	
	
		
\begin{Verbatim}[frame=single]
Real *8 Function f (x,y)
  Implicit None

  Real *8 :: x,y 

  f= 1.0 + sin(x*y)
End Function f

Program Main

  Implicit None
  Real *8 :: Xin=0.25, Yin=2.0 , c, f

  c=f(Xin,Yin)
  write (*,*) 'f(Xin,Yin)=' , c

End Program Main

\end{Verbatim}



\subsection*{Subroutine}
Y un ejemplo del manejo de subrutinas. En el c\'odigo, se escribe la rutina a realizar antes de iniciar el programa, la rutina se llama al momento de requerirla especificando los par\'ametros.\\ \\
\adjincludegraphics[scale=0.67,trim={0 0 {.5\width} 0},clip]{Subroutine}


	
\begin{Verbatim}[frame=single]
Subroutine g(x,y,ans1,ans2)
  Implicit None
  Real (8) :: x,y,ans1,ans2

  ans1=sin(x*y) + 1 
  ans2=ans1**2

End Subroutine g

Program Main_program

  Implicit None
  Real *8 :: Xin=0.25, Yin=2.0, Gout1, Gout2

  call g(Xin,Yin,Gout1,Gout2)
  write (*,*) 'The answers are:' , Gout1, Gout2

End Program Main_program
\end{Verbatim}


\end{document}